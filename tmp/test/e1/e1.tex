\documentclass[prova]{braun}

\title{Prova de Etapa 1}

\subject{Química}

\author{Gabriel Braun}

\date{13/10/2022}

\begin{document}
\maketitle

\paragraph{Dados}

\paragraph{Elementos}
\begin{center}
\PTableRow{C, H, O, Ne, U}
\end{center}

\hspace{2em}

\begin{problem}[
	points={10}
]
\textbf{Assinale} a alternativa que mais se aproxima da energia liberada
por {\(\qty{5}{g}\)} de sódio em uma lâmpada que produz luz amarela com
comprimento de onda {\(\qty{590}{nm}\)}.

\begin{choices}
\item {\(\qty{100}{kJ}\)}

\item {\(\qty{200}{kJ}\)}

\item {\(\qty{300}{kJ}\)}

\item {\(\qty{400}{kJ}\)}

\item {\(\qty{500}{kJ}\)}

\end{choices}

\end{problem}

\newpage

\begin{problem}[
	points={10}
]
Uma amostra contendo {\(\qty{0,1}{mol}\)} de nitrato de cálcio,
{\(\qty{0,1}{mol}\)} de nitrato de bário e {\(\qty{0,15}{mol}\)} de
sulfato de sódio foram adicionados em {\(\qty{600}{mL}\)} de água
destilada.

\begin{enumerate}
\def\labelenumi{\alph{enumi}.}
\tightlist
\item
  \textbf{Determine} a concentração de todas as espécies em solução no
  equilíbrio.
\item
  \textbf{Determine} outra coisa.
\end{enumerate}

Dados

\begin{itemize}
\tightlist
\item
  Produto de solubilidade do sulfato de cálcio
  {\(K_\mathrm{ps}(\ce{CaSO4}) = \num{1e-10}\)}
\item
  Produto de solubilidade do sulfato de bário
  {\(K_\mathrm{ps}(\ce{BaSO4}) = \num{2e-20}\)}
\end{itemize}

\end{problem}

\end{document}
